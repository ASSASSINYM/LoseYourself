\documentclass[11pt]{article}
\usepackage[top=1in, bottom=1in, left=1in, right=1in]{geometry}
\title{The Uses of  Socials Statistics}
\author{Kevin Macharia(I56/87243/2017) \\ Assignment One \\ SST602: Modelling and Analysis of Social Data}
\date{\today}
\begin{document}
\maketitle
\paragraph{Introduction:}Statistics is the art of collecting, summarising, presenting and interpreting data about a population of interest. Primarily social statistics involves studying human behavior in a social environment. \\
In social statistics the main areas of study include:-

\begin{itemize}
\item Educational statistics
\item Housing statistics
\item Demoraphy statistics
\item Crime statistics
\item Social security statistics
\item Health statistics
\item Labour statistics
\item Environmental statistics
\item Food and Agricultural statistics  
\end{itemize}
\textbf{Social statistics use in demography:} High dynamic of demographic processes is observed in particular in the beginning of transition period. This dynamics is caused both by political and economic reasons. In some regions dramatic changes of demographic situation is observed.For instance we can be confronted with the following specific phenomena:-
\begin{itemize}
\item Significant dynamic changes of basic vital indicators in first years of transition: decrease of rate of birth, overmortality of men, rapid decrease of rate of marriages, increase of rate of divorces, changes life expectancy, changes of fertility rate etc.)
\item Political migrations: migrations caused by political reasons (e.g. migrations stimulated by introducing new laws on citizenship, on official national language etc.)
\item Economic migrations: economic disturbances in first years of transition have caused economic international migrations: permanent, long term, short term and periodical.
\item Refugees: permanent, long term, short term and periodical migrations caused by military actions.
\item Ethnic re-emigration: after the creation of new independent states ethnic migrations of people to their ethnic states.
\item Migrations caused by disasters: natural, ecological.
\end{itemize}
\noindent Social Statistics in this case is used to give vital information on the demographic structure, population size and other key indicators which can be grouped based on civil status, ethnicity, religion,
education, profession, and economic activity - employment, housing conditions.This in turn enables the government in planning and resource allocation based on the social data.\\

\noindent\textbf{Social statistics use in housing:} The specific phenomena here includes the following:
\begin{itemize}
\item Changes in housing conditions of population caused by market economy and cuts of government subsidies (significant increase of the share of rents for flats in family budgets).
\item Substance of housing: in some regions political disturbances (including military actions) cause losses in the quantity and quality substance ofhousing.
\item Regionally concentrated migrations caused by housing conditions.
\item Consequences of demographic process (rate of births, marriages, divorces, migrations) on housing situation.
\item Homelessness.
\item Commercialization and privatization of housing and its social and economic consequences for households (change of income for disposal, negative income for disposal etc.)
\end{itemize}  
\noindent This is used to estimate the homeless population by regions, with special reference to urban
areas and big cities, social structure of homeless people, Level and dynamics of rents for flats and their share in incomes of households and development of dwelling market (prices of flats, rents for flats ratio of the price to the average income and salary). The data are necessary to evaluate the mobility
of labor force in the country. \\

\noindent\textbf{Social statistics use in labour:} The specific phenomena here includes the following:
\begin{itemize}
\item Shadow and ill-registered employment and self-employment.
\item Incidental, unstable, short time and part time employment.
\item Unemployment generated by the processes of the restructuring of industry.
\item Many, relatively small, relatively isolated, autonomous local labor markets, i.a. because of underdevelopment of modern commuting infrastructure, high costs of flats on free market and the structure of urbanization of the country ("onefactory dependent" towns).
\item Underdevelopment of administrative infrastructure organizing labor markets and unemployment.
\end{itemize}  
\noindent This is used to generate a statistical map of local labor markets to help the governments to evaluate the situation on regional and local labor markets and to chose most effective measures to preventing and fighting against unemployment and its social consequences,validation and quality control of data on employment and jobs collected directly form businesses  and estimate the costs of labor for employers. \\

\noindent\textbf{Social statistics use in health:} The specific phenomena here includes the following:
\begin{itemize}
\item Fundamental changes (implemented or under preparation) of public health care systems, from public services provided by government to market - driven services.
\item Changes of the system of financing of health care: from government budget to health insurance separated from government budget and direct financing of health services by households.
\item Changes in health status of different classes of population, with special reference to
  \begin{itemize}
  \item Social groups around and below poverty line.
  \item The unemployed.
  \item People who do not benefit from health insurance (secondary effect of economic and social polarization
    of population).
  \end{itemize}
\end{itemize}
\noindent This is very useful in  compiling statistics on health services.It is needed for proper budgetary policy of governments, reform of health insurance systems and for proper transition of national health systems. It also gives insight in expenditures for health services and government subsidies addressed to the population
with low income, to social groups below poverty line and regions of disasters. \\

\noindent\textbf{Social statistics use in education:} The specific phenomena here includes the following:

\begin{itemize}
\item Institutional changes of national systems of education as the entire component of transition process, adjustment of education system to market - driven economy and to national tradition, especially in new independent states (e.g. religious and ethnic schools).
\item Organizational changes of education system to new organization of regional and local self-government.
\item Changes in financing the education system: relative decrease of financing from central government budget, increase of financing by local self - governments, social organizations, including NGO`s, churches and other religious organizations.
\item Gaps between the profiles of education and skills of population and the needs of market driven economy in transition. The re-training of large groups of workers in restructuring branches of the economy.
\end{itemize}
\noindent Data on education profiles and the profiles of jobs to support the elaboration of programs of re-training of workers, to adopt their skills to the situation on local and national labor market. \\

\noindent\textbf{Social statistics use in environment:} The specific phenomena here includes the following:
\begin{itemize}
\item Environmental "heritage" of centrally planned economy - devastation of environment:
\item Concentration of polluting industries in selected regions
\item Low level of environment protection facilities
\item Large areas durably or permanently devastated (some regions excluded from economic and social use forever)
\item Economic and social degradation of polluted areas: in the process of transition the
  most polluting branches should be restructures and their production should be.
\end{itemize}
\noindent This is used to give indicators of the environment, level of ecological degradation which may be Kinds of pollution or degradation, Branch structure of the economy and Structure of land.This is also used to determine the population living on ecologically wasted areas by Level of ecological degradation, Kinds of pollution or degradation, Structure of population living in regions (age, sex, economic activity status) and Special indicators characterizing standard of life and health conditions of population in polluted areas. \\

\noindent\textbf{Social statistics use in Social security, crime and justice:} The specific phenomena here includes the following:
\begin{itemize}
\item Processes of transition in some regions are accompanied by dramatic political and social events (military actions, ethnic disturbances, connected with violence of human and civil rights).
\item Official statistics should deliver basic social indicators these processes, on population living in the areas of military, social or ethnic disturbances, displacement of population and social consequences of these events and processes.
\item Official statistics helps the governments to evaluate and forecast social consequences
of those events for population living in the areas, to estimate the number and structure
of population and to organize respective measures.
\end{itemize}
\noindent This is used to estimate the  population living in the areas of conflicts, to quantify the
number and kind of human rights violence and crimes and  population suffering because
of security problems. \\

\noindent\textbf{Social statistics use in Nutrition:} The specific phenomena here includes the following:
\begin{itemize}
\item Polarization of the quality of nutrition of households by incomes and by incomes for
  disposal.
\item Changes in nutrition caused by significant changes of structure of prices (cuts of
government subsidies to basic food products have changed the level and structure of
prices and consumption preferences).
\item Regional and local diversification of food and nutrition (regions of social or economic
  disturbances etc.).
\item Nutrition of households with low income for disposal and bellow national poverty line.
\item Aggregated data on expenditures for nutrition (e.g. consumption compiled on the basis
on retail sales of per capita) may not represent real nutrition standards of households.  
\end{itemize}
\noindent This is used in the analysis of nutrition of population with low income for disposal, social groups below poverty line, regions of disasters. This is also used for proper nutritional planning for households.
\end{document}
